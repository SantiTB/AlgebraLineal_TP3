\item Sean $V$ y $W$ dos $\K$-EV. Supongamos que $V$ es de dimensión finita y sea $T:V\to W$ una transformación lineal. Probar que:
    \begin{enumerate}
        \item $T$ es inyectiva si, y sólo si, $dim(Im(T))=dim(V)$.
            \begin{mdframed}[style=s]
                \begin{itemize}
                    \item $(\Rightarrow)$\\
                        Si $T$ es inyectiva, entonces $N(T)=\{0\}\to dim(N(T))=0$. Por el teorema de las dimensiones $dim(V)=dim(Im(T))+dim(N(T))$
                        \[\to dim(V)=dim(Im(T))\]
                    \item $(\Leftarrow)$\\
                        Los pasos anteriores valen para ambas direcciones.
                \end{itemize}
            \end{mdframed}
        \item $T$ es suryectiva si, y sólo si, $dim(Im(T))=dim(W)$.
            \begin{mdframed}[style=s]
                \begin{itemize}
                    \item $(\Rightarrow)$\\
                        $T$ es suryectiva $\to W=Im(T)\to dim(W)=dim(Im(T))$
                    \item $(\Leftarrow)$\\
                        $dim(Im(T))=dim(W)$, como $Im(T)$ es subespacio de $W$, $Im(T)=W\to T$ es suryectiva.
                \end{itemize}
            \end{mdframed}
        \item Supongamos que $dim(V)=dim(W)$. Luego, $N(T)=\{0\}$ si, y sólo si, $T$ es suryectiva.
            \begin{mdframed}[style=s]
                $N(T)=\{0\}\Leftrightarrow dim(N(T))=0\Leftrightarrow dim(V)=dim(W)=dim(Im(T))\Leftrightarrow W=Im(T)\Leftrightarrow T$ es suryectiva.
            \end{mdframed}
    \end{enumerate}
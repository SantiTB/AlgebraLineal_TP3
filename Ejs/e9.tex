\item Sean $R_A:\R^3\to\R^3$ y $L_A:\R^{3\times1}\to\R^{3\times1}$, dados por\[R_A(x_1,x_2,x_3)=(x_1,x_2,x_3)A\quad \text{y}\quad L_A\begin{pmatrix}
        x_1\\x_2\\x_3
    \end{pmatrix}=A\begin{pmatrix}
        x_1\\x_2\\x_3
    \end{pmatrix},\]donde\[A=\begin{pmatrix}
        -1&0&2\\5&4&2\\-4&-3&2
    \end{pmatrix}\]\begin{enumerate}
        \item Probar que $R_A\in L(\R^3)$ y $L_A\in L(\R^{3\times1})$.
            \begin{mdframed}[style=s]
                \begin{itemize}
                    \item $R_A$\\
                        Dado $u,v\in\R^3,\sigma\in\R$
                        \begin{align*}
                            R_A(\sigma u+v)&=(\sigma u+v)A&&\text{Definición }R_A\\
                            &=(\sigma u)A+vA&&\text{Distributividad de matrices}\\
                            &=\sigma uA+vA&&\text{Producto por escalar en matrices}\\
                            &=\sigma R_Au+R_Av&&\text{Definición }R_A
                        \end{align*}
                        Por lo tanto, $R_A\in L(\R^3)$
                    \item $L_A$
                        Sean $B,C\in\R^{3\times1},\sigma\in\R$
                        \begin{align*}
                            L_A(\sigma B+C)&=A(\sigma B+C)&&\text{Definición }L_A\\
                            &=A\sigma B+AC&&\text{Distributividad de matrices}\\
                            &=\sigma AB+AC&&\text{Producto por escalar}\\
                            &=\sigma L_AB+L_AC&&\text{Definición }L_A
                        \end{align*}
                        Por lo tanto, $L_A\in L(\R^{3\times1})$
                \end{itemize}
            \end{mdframed}
        \item Hallar el núcleo y la imagen de $R_A$ y $L_A$. Analizar, en ambos casos, inyectividad, suryectividad y biyectividad.
            \begin{mdframed}[style=s]
                \begin{itemize}
                    \item $R_A$\\
                        $N(R_A)=\{v\in\R^3:R_A(v)=\vec{0}\}$\[(x,y,z)\begin{pmatrix}
                            -1&0&2\\5&4&2\\-4&-3&2
                        \end{pmatrix}=(0,0,0)\to\begin{cases}
                            -x+2z=0\\
                            5x+4y+2z=0\\
                            -4x-3y+2z=0
                        \end{cases}\to x=y=z=0\]
                        $N(T)=\{(0,0,0)\}$, entonces es inyectiva, con lo cual es un monomorfismo. Como $R_A\in L(\R^3)$, es un epimorfismo y un isomorfismo. Por lo tanto, es suryectiva y biyectiva.
                    \item $L_A$\\
                        $N(L_A)=\{B\in\R^{3\times1}:L_A(B)=0\}$\[\begin{pmatrix}
                            -1&0&2\\5&4&2\\-4&-3&2
                        \end{pmatrix}\begin{pmatrix}
                            x\\y\\z
                        \end{pmatrix}=\begin{pmatrix}
                            0\\0\\0
                        \end{pmatrix}\to\begin{cases}
                            -x+2z=0\\
                            5x+4y+2z=0\\
                            -4x-3y+2z=0
                        \end{cases}\]
                        Tengo el mismo sistema que antes, con lo cual se van a cumplir las mismas propiedades.
                \end{itemize}
            \end{mdframed}
    \end{enumerate}